%%
%% This is file 'preface.tex'
%% It is included by hhuthesis-example.tex for hhuthesis.
%%
%% Copyright(C) 2020, Wenhan Cao
%% College of Water Conservancy and Hydropower Engineering, Hohai University.
%%
%% Version:v1.0.0
%% Last update: July 19th, 2020.
%%
%% Home Page of the Project: https://github.com/caowenhan/thesis
%%
%% This file may be distributed and / or modified under the conditions of the
%% LaTeX Project Public License, either version 1.3c of this license or (at your
%% option) any later version. The latest version of this license is in:
%%
%% http://www.latex-project.org/lppl.txt
%%
%% and version 1.3c or later is part of all distributions of LaTeX version
%% 2008/05/04 or later.
%%
\begin{preface}
我国修建了大量的混凝土坝工程,这些水利工程在防洪、灌溉和发电、通航等方面发挥了巨大的作用。然而,由于我国具有独特的水资源分布特征,有相当一部分混凝土坝修建于高海拔或高纬度的东北、西北等寒冷地区,这些地区普遍存在昼夜温差大、冰冻周期长和年际冻融次数多等特点,容易引发混凝土坝出现冻融与溶蚀等典型渗水病害,导致大坝服役性能衰退,给工程安全造成威胁。因此,科学地分析寒冷地区混凝土坝在渗水病害影响下工作性态演化机制,并对其服役性态进行客观评估,已成为研究的热点。\par
针对上述问题,综合运用数学方法、力学理论、坝工知识和计算机技术,从微观、细观、宏观和整体性态变化及安全角度,系统开展了寒冷地区渗水病害影响下混凝土坝服役性态多尺度分析方法研究,取得以下主要创新性成果:	
\begin{enumerate}
	\item[(1)] 针对坝体混凝土材料多尺度多物相复合属性特征,引入三维微观水化模型,研究了坝体混凝土微观结构特性;采用静水压力与结晶压力复合作用模型,分析了坝体混凝土微观冻融力学性能演化规律;提出水泥基材料析钙溶蚀模型,并采用随机溶蚀算法,探究了坝体混凝土微观溶蚀力学性能演化特性。
	\item[(2)] 分析了砂浆及混凝土材料多尺度物相特征,构建了坝体混凝土材料力学多尺度递进分析表征模型;通过引入数学渐进均匀化理论,以连续损伤力学为支撑,基于多尺度能量积分方法,提出了坝体混凝土材料在冻融和溶蚀两种典型渗水病害影响下力学性能多尺度递进分析模型。	
\end{enumerate}
	
\end{preface}